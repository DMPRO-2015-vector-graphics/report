\chapter{System Overview}

\section{Architectual structure}

\begin{figure}[h!]
    \includegraphics[width=\linewidth]{images/system-overview.png}
    \caption{A logical overview of the \vthreek architecture}
    \label{fig:system-overview}
\end{figure}

Figure \ref{fig:system-overview} shows how the major building blocks of the \vthreek architecture come together on a conceptual level.
Separation of instruction and data memory makes it a Harvard-like architecture.
The processor reads from instruction memory, executes instructions and adds/updates vector primitives in the scene memory.
Preprocessing these primitives for output is done by the output modules.
For the HDMI output, this involves rasterizing each primitive and maintaining a framebuffer.
The oscilloscope output is generated by serializing primitives onto two DACs.


\section{Output modules}

\subsection{Raster output}

This output uses HDMI to output a rasterized image of the internal vector representation.

\subsection{DAC output}

The primary output of the processor is the analogue one. The DACs are driven from the FPGA over a serial interface. The DACs need a clock, data, and sync signal to operate. The clock and sync signal is shared between the DACs so they operate synchronously. 

The parallel to serial converter takes a 32-bit integer as input which has 16 bit for x and y components of a point that should be sent to the DACs. The converter then splits this integer into its x and y components and clocks them out synchronously to their respective DACs.