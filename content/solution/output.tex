\chapter{Output Modules}
\label{chap:Output}

Generating visual output is an important requirement of the system.
This section will describe how analogue and HDMI output was realized.
Output modules are almost completely separated from the main processor, and produce output by reading from the scene memory.

\section{Analogue Output}

\begin{figure}[h!]
    \includegraphics[width=\linewidth]{images/dac-output.png}
    \caption{\gls{rtl} sketch of the \vthreek \gls{dac} output module.}
    \label{fig:dac-output}
\end{figure}

The primary output of the processor is the analogue output.
This output is driven by two \gls{dac}s that convert digital data stored in the scene memory to voltages that represent x and y coordinates.
The \gls{dac}s used are 16-bit, which results in an effective resolution of 65536 by 65536 without taking into account the noise on the output voltage.
This is elaborated on in Section~\ref{sec:noise}.
These \gls{dac}s have a serial interface and needs 3 signals each to operate; data, clock, and a sync signal.
In order to set an output value on a \gls{dac}, the sync signal needs to be pulled from high to low.
Eight control bits are then transmitted followed by 16 data bits.
After this, the sync signal needs to be pulled high again in order to set the output value.
The sync signal can be pulled high before this in order to abort the operation.
Clock and sync signals are shared between the \gls{dac}s in order to make them change their output synchronously.

\subsection{Operation}

The output module sends an address to the scene memory and gets a primitive back.
There is also another input that keeps track of the number of primitives so the module can restart without reading unused primitives.
The primitive is then decoded and forwarded to the appropriate module.
If a valid primitive is not able to be decoded, the read address is incremented and a new data is read from the scene memory.
In the case where a valid primitive is found, it is forwarded to the serializer.
The respective serializer will then calculate a fixed number of points along the line or curve described by the primitive data.
A fixed number of points was chosen because there was no time to implement any functionality for calculating the lengths of lines or curves accurately.
This will cause shorter lines and curves to appear brighter on the oscilloscope.

The actual calculation of the points is done by using the equations described in section \ref{sec:bezier}.
However since there was no floating point unit, a number between 0 and 1023 was used instead of a number between 0 and 1, that was then left shifted by 5.
\(t\) then becomes \(t = i \ll 5\), (\(0 \leq i < 1024\)).
After every multiplication the result is then right shifted by 15.
This results in a limit of 1024 different points for a given primitive.

The calculation for points for every type of primitive is largely the same, however the different number of points per primitive needs to multiplied by a different factor.
The equations for these can be found in Section \ref{sec:bezier}

\section{Raster Output}

This output uses \gls{hdmi} to output a rasterized image of the internal vector representation.
In order to generate the rasterized frame the x and y output from the vector output described in the previous section would be written into a frame-buffer.
This buffer can then be scaled to the desired output resolution and then sent over a regular HDMI connection.

Raster output was added as a medium priority requirement \ref{sec:requirements}, to be used as a backup solution if the group was unable to implement analogue output.
Since the latter was implemented successfully, not much time was spent on implementing HDMI output.
