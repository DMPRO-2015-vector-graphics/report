\chapter{Discussion}

During the design process, the group faced many issues. This chapter discusses all the significant obstacles on the way, and how the group managed to fix or work around them. Additionaly, possible improvements, which were not implemented due to time constraints, are also addressed to give a better understanding of the system.

\section{Previous Projects}
During the initial planning the group read previous computer project reports,


\section{Architecture Design}
Before any actual coding and soldering the group spent time planning the architecture.
TODO: Vi burde hatt floating point for å gjøre vektorutregninger.

\section{Budget}
TODO: update, wrong numbers

In the assignment text the group was given a budget of 10,000 NOK.
At the end of the project the total cost of materials was 9,770 NOK (without shipping costs), which is within the given budget.
The full list of material expenses is listed in Appendix \ref{tbl:material-cost}.

A few of the component orders were wrong and the group had to order new components.
With better component planning this could have been avoided.

\section{Oscillator}
To acheive a high clock frequency on the \gls{fpga}, a 100 MHz oscillator was added to the board.
Using it as the main clock signal, the \gls{fpga} was not able to produce correct output.
An oscilloscope was used to measure the oscillator output on a header.
If a hand was placed in close proximity of the board, the oscillator signal was significantly affected.
This problem was solved by using the 48 MHz oscillator instead.

\section{Drawing Artefacts}
\label{discussion:artefacts}
The drawing artefacts explained in Section \ref{results:artefacts} are possible to remove.
If a third input value for intensity were to be added, one could turn the electron beam off when moving the beam between primitive.

The other artefact mentioned, is the dotted lines.
This could be removed, by adding analogue circuitry between the DAC output and the BNC connectors.
It is possible to slew the electron beam from one point to the next, by analogue integration, to archive a smooth transition between points.
An example of such circuitry has been implemented in the 80's in a similar project \cite{vector-graphic-crt}.


If \vthreek were to support an intensity value in addition to x and y, the architecture and PCB require changes.
