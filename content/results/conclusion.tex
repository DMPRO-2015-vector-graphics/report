\chapter{Conclusion}
The group have designed and implemented a multi cycle vector graphics processor.
It is realized on a custom made \gls{pcb}, with an \gls{fpga} from Xilinx as the main processing unit.
A program written in low level code using a self made \gls{isa} executes on the processor, generating the basis for vector graphics.
The graphics are displayed on the vector display of an old Tektronix oscilloscope.

The system meets the primary requirements defined by the group in the introductory phase. 
Some advanced requirements were not met due to time constraints, but are possible to implement given enough time. 

The computer design project gave all group members valuable insight into how to create a computer from start to finish.
While i definitely being a challenging project, the group has produced a working product and learned a great amount during the process.
A lot of the 'magic' behind computer design has now been removed.

\section{The Future of Vector Graphics}

Working with the vector display, the group noticed that vector displays today are an outdated technology. 
While raster displays have moved on to \gls{lcd} and \gls{led} screens, all vector displays are still using outdated and potentially harmful CRT technology.
Should vector graphics be efficient to use in the context of modern computing, a similar leap forward in vector display technology is a necessity.
