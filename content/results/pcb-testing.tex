\section{Physical PCB and Components testing}
Most the components were tested independently, and then tested as a whole on the \gls{pcb}.
Some major components were impossible to test without soldering on the \gls{pcb} though.
Examples were the \gls{fpga} and \gls{mcu}, because of their tiny ball grid pins.

\subsection{Testing of Physical PCB}
The team had to find and fix all errors and faults with the \gls{pcb} design and components, that were a hazard to the desired resulting system.
The team had some direct initial tests of the \gls{pcb} itself,
but afterwards it was indirectly tested, when testing the \gls{fpga}, \gls{mcu}, \gls{dac}s etc.
It was in these indirect tests, most errors were discovered.

Testing was done by using a multimeter and oscilloscopes.
Using a multimeter was the best way to verify if a signal was having the expected voltage level, and if there was any current flow at all.

Oscilloscopes were regularly used to analyze frequency and accuracy of the oscillators.
Note that the oscilloscope which was used as the vector display was not part of this testing.
These were modern oscilloscopes, which had raster screens and operating systems installed.
