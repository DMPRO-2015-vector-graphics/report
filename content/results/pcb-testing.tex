\section{Physical PCB and Components testing}

// TODO: Restructure this, some of it belongs in results, some in discussion.

When we had received the PCB and the components, we had to make sure that every component worked properly independent of the PCB, then testing if they worked on the PCB. Many major components were impossible to test without soldering on the PCB though. Examples were the FPGA and MCU, because of their tiny ball grid pins.

\subsection{Testing of Physical PCB}
We had to find and fix all errors and faults with our PCB design and components, that were a hazard to our desired resulting system.  
We had some direct initial tests of the PCB itself, but afterwards it was indirectly tested, when testing the FPGA, MCU, DACs etc. It was in these indirect tests, most errors were discovered.

Testing was done by using a multimeter and oscilloscopes. Using a multimeter was the best way to verify if a signal was having the expected voltage level, and if there was any current flow at all.

Oscilloscopes were handy for testing our oscillators, to check the frequency and accuracy. Note that the oscilloscope which was used as our vector display was not part of this testing. These were modern oscilloscopes, which had raster screens and operating systems installed.

