\section{Oscillator}
To acheive a high clock frequency on the FPGA a 100 MHz oscillator was added to the board.
However each time the 100 MHz oscillator was used as the main clock signal,
the FPGA was not able to produce correct output.
An oscilloscope was used to measure the oscillator output on a header.
If a hand was placed in close proximity of the board the oscillator signal was significantly affected.
The measured signal was also too low peak to peak, for the FPGA to be used as a clock signal.
Placing the 100 MHz oscillator closer to the FPGA would most likely have removed it's sensitivity to noise.
