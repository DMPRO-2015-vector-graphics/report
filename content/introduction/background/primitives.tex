\section{Primitives}
The computer arcitechture supports 3 different primitives, lines, quadratic Bézier curves and qubic Bézier curves
Each primitive type are identified by the 8 first bits.
To store a position on the screen 32 bits is required, 16 bit for x and 16 bit for y.
A line is represented by a start point and an end point.
To represent a Quadratic Bézier curve a total of three points are needed, a start point, a control point and an end point.
The last primitive a Qubic Bézier curve, is equal to a Quadratic Bézier, but have one extra control point.

Table \ref{tbl:primitives} shows how each primitive is stored in memory.
A line start point is \(x_0 \) and \(y_0 \), and end point is \(x_1 \) \(y_1 \)

With the basic primitives line, quadratic Bézier and qubic Bézier all other shapes can be constructed.
E.g a rectangle is constructed by adding four lines together.

\begin{table}[h]
    \label{tbl:primitives}
    \centering
    \begin{tabular}{|l|l|l|l|l|l|l|l|l|l|}
    \hline
    8                & 16  & 16  & 16  & 16  & 16  & 16  & 16  & 16  & Total \\ \hline
    Line             & \(x_0 \)  & \(y_0 \)  & \(x_1 \)  & \(y_0 \)  & ~   & ~   & ~   & ~   & 72    \\ \hline
    Quadratic Bézier & \(p_0^x \) & \(p_0^y \) & \(p_1^x \) & \(p_1^y \) & \(p_2^x \) & \(p_2^y \) & ~   & ~   & 104   \\ \hline
    Qubic Bézier     & \(p_0^x \) & \(p_0^y \) & \(p_1^x \) & \(p_1^y \) & \(p_2^x \) & \(p_2^y \) & \(p_3^x \) & \(p_3^y \) & 136   \\ \hline
    \end{tabular}
    \caption{The table above shows how each primitive is stored in memory}
\end{table}
