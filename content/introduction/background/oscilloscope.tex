\subsection{Oscilloscope}
An oscilloscope is a popular lab tool used to output voltage as a function of time.
Modern oscilloscopes often use thin film transistor (TFT) displays and rasterize output. 
However, older models include a CRT vector display.

To draw on such an oscilloscope, the device must support at least two input channels, and the ability to draw in X-Y mode.
This means that two of the input channels (usually Ch1 and Ch2) governs the electron ray deflectors, which are controlled through electrostatics.
If the channels are provided with a constant voltage, a dot will be displayed on the screen - in contrast to normal mode, where a line would be drawn. // TODO: What is 'normal' mode? Varying voltage?

The voltage on channel one will normally decide the beam's horizontal X position, and channel two its vertical Y position.
To draw a line on the oscilloscope, frequent changes to the voltage of one or both channels is necessary to make the image remain on the screen. // For our osc this is true, but other types may be different
Changing the voltage on only one channel will produce a horizontal or vertical line.
