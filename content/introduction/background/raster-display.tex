\subsection{Raster Displays}
There exist a variety of raster display technologies.
However, they have a few common attributes.
Raster graphics use a technique called frame buffer, which is a register used to hold the current image (frame). // register correct?
When a new image has been produced by the computer, it is promptly loaded into the frame buffer.
The image is then scanned (drawn) from the rows of pixels (called scanlines) in the frame buffer, until it is displayed on the screen.

The scan is often done from top to bottom of the image; this is called a progressive scan.
An interlaced scan is another scan algorithm where the even and odd scanlines are scanned alternately, often resulting in higher frame rates but increased screen flicker.