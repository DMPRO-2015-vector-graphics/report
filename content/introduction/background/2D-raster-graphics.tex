\subsection{2D Raster Graphics}
Raster graphics are also known as bitmap graphics, and consist of a two-dimensional array of discrete values, where each discrete value is used to describe the color of the pixel at a given position in the array.
A bitmap image is often represented in the RGB color scheme - that is where you define different levels of red, green and blue for each pixel until you are able to render the correct color.

To display a full image, the computer needs to store \( (width \times height) \times RGB \) bits.
Today, \gls{rgb} usually contains 3 bytes, one byte for each color.
Modern computer screens commonly have resolutions of \(1920 \times 1080\) pixels (Full HD) or higher.
With Full HD resolution, a single image requires about 6 MB to be stored.
There exists method to compress images, such as different file formats.
However, raster images still use a significant amount of storage space.

A benefit of using raster graphics is that images of real life can be represented in great detail, given a high enough pixel count.
If a higher level of detail in an image is needed, the image can be redrawn, regenerated or photographed again in a larger resolution.
A common use case is storing photographs and images used on the web.
In today's computing world, raster graphics is the dominating paradigm, and almost all digital 2D graphics you encounter are stored in a rasterized format, such as \texttt{.jpeg}, \texttt{.png} and \texttt{.gif}.
