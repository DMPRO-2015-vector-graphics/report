\subsection{2D Raster Graphics}
Raster graphics are also known as bitmap graphics, and consist of a two-dimensional array of discrete values, where each discrete value is used to describe the color of the pixel at a given position in the array.
A bitmap image is often represented in the RGB color scheme - that is where you define different levels of red, green and blue for each pixel until you are able to render the correct color.

To display a full image, the computer needs to store \( (width \times height) \times RGB \) bits.
Today, \gls{rgb} usually contains 3 bytes, one byte for each color, and a resolution of \(1920 \times 1080\) pixels (Full HD) is becoming the standard.
With Full HD resolution, a single image therefore requires about 6 MB to be stored.
There exists method to compress images, such as different file formats.
However, raster images still uses a significant amount of storage space.

A benefit by using raster graphics is that it can be used to store any image, and still keep almost any level of detail intact.
If a higher level of detail in an images is needed, the pixel count may be increased, but the original image has to be rasterized over again.
A common use case is storing photographs and images used on the web.
In today's computing world, raster graphics is the dominating paradigm, and almost all digital 2D graphics you encounter are stored in a rasterized format, such as \texttt{.jpeg}, \texttt{.png} and \texttt{.gif}.
