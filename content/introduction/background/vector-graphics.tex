\subsection{Vector Graphics}
Vector graphics is a 

With vector graphics, mathematical expressions are used to describe an image.
The image, or the scene, is made up of primitives that describe paths and curves in order to create more complex shapes.

By relying on mathematical expressions instead of pixel values relative to screen resolution, vector graphics can be described as resolution independent. This is often conceptualized in the form of scaling: While scaling a rasterized image will result in a blurry image, scaling a vectorized image will result in the mathematical expressions being recalculated, and the image will remain sharp.

Vector graphics are predominantly used in 2D graphics, and is called vector graphics in order to distinguish from 2D raster graphics.
In 3D graphics the internal representation of the world is usually described with vector primitives (points that form triangle surfaces for example).
TODO: Some citation.

Some computer font types, like TrueType uses vector graphics and more specifically Bézier curves to represent glyphs\cite{truetype}.

Vector graphics are seldom used in today's computer realm, mainly due to monitor disadvatages described later in this chapter. Another reason may be that raster graphics are just easier to comprehend and work with.