\chapter{TDT4295 - Computer Design Project}

The Computer Design Project is held at NTNU every fall.
It is a large, project-based subject in which students create a working computing platform, more or less from scratch.
This year saw high enough participation that two assignments were presented, and two groups formed around these.
This report details the work done and solution implemented by the vector graphics processor group.

\section{Assignment}

This years assignment focuses on graphics, exploring both of the traditional ways of representing and processing graphics; raster-based and vector-based.
Two distinct assignment texts were presented by the course staff, one focusing on raster-based graphics, the other on vector graphics.
The following is a verbatim copy of the vector graphics assignment text \cite{assignment-text}.

\subsection{Assignment requirements - A Vector Graphics Processor}

Image generation using vector graphics as the core method is a powerful and scalable way of generating image data.
Vector graphics represents images at a higher abstraction level than single-pixels.
A vector graphics processor can make use of drawing instructions to produce images, which is a task well suited for hardware acceleration [3 // TODO: add references].
Parallelization with multiple cores and/or at the instruction level are architectural possibilities that can be exploited to design a specialized processor.
The task is to design and implement a processor for producing vector graphics.
Figure 2 illustrates a vector display which takes in vector drawing commands (instead of a stream of pixels) as input.

//TODO: add image?

\subsection{Additional requirements}

Your processor will be implemented on an FPGA, and you are free to choose how to realize your computer architecture.
Studying the architecture of general multi-core processors [6], and parallel machines options [4, 5, 6, 7] can be a good starting point.


The task should also include a suitable application that can process/produce graphical data.
The output is to be displayed in order to demonstrate the processor.

The unit must utilize a Silicon Labs EFM32 series microcontroller (to act as an I/O processor) and a Xilinx FPGA (to implement your architecture on).
The budget is 10.000 NOK per group, which must cover components and PCB production.
The unit design must adhere to the limits set by the course staff at any given time.
Deadlines are given in a separate time schedule.

\subsection{A Vector Graphics Computer Architecture}

// TODO: Bedre tittel?

The group chose to make a general purpose computer with support for producing vector graphics.
The computer would read instructions from memory, execute them on a processor core and render vector based scenes to different outputs.
Since most commonly available display devices expect raster based input, an oscilloscope was chosen as the primary output device, HDMI as a secondary.
This means that in addition to a processor core and SRAM interfaces, output preprocessors for both output modes would have to be implemented.
In the case of HDMI, a rasterizer, and for oscilloscope output, a serializer feeding data to DACs.

As a whole, the system should be assembled on a custom PCB, with the microcontroller serving as an I/O-unit while the main processor architecture and output-modules should be implemented on the FPGA.
The microcontroller would be tasked with loading a program into memory on boot, so that the processor has something to work with.

