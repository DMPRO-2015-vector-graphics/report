\chapter{Background and Theory}

Vector graphics have been popular in the computer industry for many years, seeing frequent use in analogue oscilloscopes, video arcade games \cite{astroids}, and as display devices for computers \cite{ibm2250}.

\section{Vector Graphics}

With vector graphics, mathematical expressions are used to describe an image.
The image, or the scene, is made up of primitives that describe paths and curves in order to create more complex shapes.

Vector graphics are predominantly used in 2D graphics, and is called vector graphics in order to distinguish from 2D raster graphics.
In 3D graphics the internal representation of the world is usually described with vector primitives (points that form triangle surfaces for example). TODO: Some citation.

Some computer font types, like TrueType uses vector graphics and more specifically Bézier curves to represent glyphs\cite{truetype}.


TODO: Write about advantages of vector graphics over raster, the background chapter should explain the motivation for solving the assignment.


\section{Vector Monitor}
A vector display or monitor, is a device that draws graphics from point to point. // TODO: As opposed to?
A vector monitor utilizes CRT (Cathode Ray Tube) technology, which contain one or more electron guns and a phosphorescent screen to view images.
The electron beam is deflected horizontally and vertically using electrostatic deflection. \cite{vector-monitor}

Unlike the CRT raster displays (old television sets or computer monitors), a vector monitor does not scan repeatedly in a fixed pattern.
Instead it draws a point based on two voltages, one for horizontal placement and one for vertical.

One of the major advantages with vector monitors, is that since they are able to draw directly from one point to another, they do not suffer from artefacts like aliasing and pixelation.
The drawback however is its inability to fill shapes in an efficient way.
Therefore, vector monitors usually only draw the outline of shapes.


\section{Drawing to an oscilloscope}
As vector monitors are no longer a easily available, the group needed to find an alternative method of displaying vector graphics, preferrably without rasterization involved.
It is possible to modify a CRT monitor, so that one can manually control the deflectors, but this is quite cumbersome.
The other method is using an oscilloscope, which gives an easy interface to these deflectors.
This section will explain how to use the oscilloscope as a vector monitor.


To be able to draw to a oscilloscope, the oscilloscope must support at least two input channels, and the ability to draw in X-Y mode.
In X-Y mode, two of the input channels (usually Ch1 and Ch2) governs the electron ray deflectors.
If the channels are provided a constant voltage, a dot will be displayed on the screen, in contrast to normal mode, where  a line would be drawn.

The voltage on channel 1 will normally decide the beam's horizontal position, and channel two, its vertical position.
To draw a line on the oscilloscope, one would need to repeatedly change the voltage of one or both channels back and forth.
Changing the voltage on only one channel will produce a horizontal or vertical line. 

